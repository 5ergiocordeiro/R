\documentclass[12pt,fleqn]{amsart}
\usepackage{enumitem}
\usepackage{xstring}
\usepackage{graphicx}
\usepackage[a4paper]{geometry}
\usepackage[portuguese,brazilian]{babel}
\usepackage[ansinew]{inputenc}
\usepackage[T1]{fontenc}

\title{Teoria Eletromagn�tica - D�vidas}
\date{}
\author{S�rgio Cordeiro}

\begin{document}
\setlength{\parskip}{1em}
\setlength{\jot}{10pt}
\setlength{\parindent}{0pt}

\newcounter{prob_num}
\setcounter{prob_num}{1}

\maketitle

\section*{Lista de Exerc�cios 1}
Parte 2: 26,45,50a
\section*{Lista de Exerc�cios 2}
Parte 1: 13b \\
Parte 3
\section*{Lista de Exerc�cios 3}
m�todo de c�lculo
Parte 3
\section*{Lista de Exerc�cios 4}
15, 17, 21
\section*{Livro Annita Macedo}
\section*{Trabalho 1}
\section*{Trabalho 2}
\section*{Outras}
C�pia da prova
C�lculo de k quando a permissividade � complexa
\end{document}
