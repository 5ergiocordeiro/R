% Primeira lista de exerc�cios de Sistemas e Sinais
% Preparado para MiKTeX 2.9. Gerar arquivo PDF com pdflatex.
\documentclass[12pt,fleqn]{amsart}
\usepackage{enumitem}
\usepackage{xstring}
\usepackage{graphicx}
\usepackage[a4paper]{geometry}
\usepackage[portuguese,brazilian]{babel}
\usepackage[ansinew]{inputenc}
\usepackage[T1]{fontenc}
\usepackage{comment}
% Fontes
\usepackage{textgreek}
\usepackage{yfonts}
\usepackage{microtype}
\usepackage{calligra}
\usepackage{lmodern}
\usepackage{bookman}
\usepackage[scaled]{helvet}
\usepackage[usenames,dvipsnames,svgnames,table]{xcolor}
\usepackage[hyphens]{url}
\usepackage{hyperref}
\usepackage{fmtcount}
% Formata��o
\usepackage{sidecap}
\usepackage{float}
\usepackage{listings}
\usepackage{matlab-prettifier}
% S�mbolos matem�ticos
\usepackage{amsmath}
\usepackage{amssymb}
\usepackage{wasysym}
\usepackage{mathtools}
\usepackage{bbm}
\usepackage{mbboard}


\title{Sistemas e Sinais - Exerc�cio I}
\date{}
\author{S�rgio Cordeiro}


\newenvironment{pergunta}{
	\par
	\trivlist
	\fontfamily{phv}
	\selectfont
	\item\arabic{prob_num}. 
	}{
	\endtrivlist
	}

\newenvironment{resposta}{
	\trivlist
    \vspace{3pt}
	\fontfamily{pbk}
	\selectfont
	\item
	}{
	\vspace{3pt}
	\noindent\makebox[\linewidth]{\rule{\paperwidth}{1pt}}
	\endtrivlist
	\stepcounter{prob_num}
	}

\newenvironment{listae}[1][]{
  \IfStrEqCase{#1}{
    {1}{\setenumerate[0]{label=\arabic*.}}
    {2}{\setenumerate[0]{label=\alph*)}}
    {3}{\setenumerate[0]{label=\roman*.}}
	{}{\setenumerate[0]{label=\arabic*)}}}
	\vspace{-5mm}
	\begin{enumerate}}{
	\end{enumerate}	
	\vspace{-5mm}}

\newenvironment{listaef}[1][]{
  \IfStrEqCase{#1}{
    {1}{\setenumerate[0]{label=\arabic*.}}
    {2}{\setenumerate[0]{label=\alph*)}}
    {3}{\setenumerate[0]{label=\roman*.}}
	{}{\setenumerate[0]{label=\arabic*)}}}
	\vspace{-5mm}
	\begin{enumerate}}{
	\end{enumerate}}

\DeclareMathAlphabet{\mathpzc}{OT1}{pzc}{m}{it}

\DeclareMathOperator{\La}{\mathcal{L}}					% Transformada de Laplace
\DeclareMathOperator{\De}{\boldsymbol{\delta}}			% Fun��o impulso
\DeclareMathOperator{\dDe}{\boldsymbol{\dot{\delta}}}	% Derivada da fun��o impulso
\newcommand{\dt}[1]{\frac{d {#1}}{dt}}				% derivada temporal
\newcommand{\ddt}[1]{\frac{d^2 {#1}}{dt^2}}			% derivada segunda temporal
\newcommand{\dtau}[1]{\frac{d {#1}}{d \tau}}		% derivada temporal
\newcommand{\ddtau}[1]{\frac{d^2 {#1}}{d \tau^2}}	% derivada segunda temporal


% Linhas divis�rias
\newcommand{\hhhlin}{\noindent\hfil\rule{0.25\textwidth}{.2pt}\hfil\newline}
\newcommand{\hhlin}{\noindent\hfil\rule{0.5\textwidth}{.4pt}\hfil\newline}
\newcommand{\hlin}{\noindent\hfil\rule{\textwidth}{.8pt}\hfil\newline}


% Perguntas
\newcommand{\perguntaa}{Mostrar que o sistema descrito pela equa��o abaixo � linear, invariante no tempo e causal:
\begin{align}
	\ddt{y} + 5 \dt{y} + 6y = 3 \dt{x} + x
\end{align}
}

\newcommand{\perguntab}{Demonstrar as propriedades \textgoth{P3} e \textgoth{P4} da Transformada de Laplace.
}


\begin{document}
\setlength{\parskip}{1em}
\setlength{\jot}{10pt}
\setlength{\parindent}{0pt}

\maketitle

\newcounter{prob_num}
\setcounter{prob_num}{1}


\begin{pergunta}
\perguntaa{}
\end{pergunta}
\begin{resposta}
a) \textbf{Linearidade} \\
Sejam $ y_1 $ e $ y_2 $ as respostas do sistema aos dois sinais de entrada $ x_1 $ e $ x_2 $, respectivamente. Nesse caso, teremos:
\begin{align}
	\ddt{y_1} + 5 \dt{y_1} + 6 y_1 = 3 \dt{x_1} + x_1 \label{Eqa1}
\end{align}
e
\begin{align}
	\ddt{y_2} + 5 \dt{y_2} + 6 y_2 = 3 \dt{x_2} + x_2 \label{Eqa2}
\end{align}
Seja $ x_3 $ uma combina��o linear de $ x_1 $ e $ x_2 $. Podemos escrever:
\begin{align}
	x_3 = a x_1 + b x_2 \qquad \Aboxed{a,b \in \mathbb{R}} \label{Eqa3}
\end{align}
A resposta $ y_3 $ do sistema ser�, ent�o:
\begin{align}
	\ddt{y_3} + 5 \dt{y_3} + 6 y_3 & = 3 \dt{x_3} + x_3 \notag \\
	& = 3 \dt{}(a x_1 + b x_2) + (a x_1 + b x_2) \notag \\
	& = 3 \dt{(a x_1)} + 3 \dt{(b x_2)} + a x_1 + b x_2 \notag \\
	& = a \left( 3 \dt{x_1} + x_1 \right) + b \left( 3 \dt{x_2} + b x_2 \right) \label{Eqa4}
\end{align}
De acordo com \ref{Eqa1} e \ref{Eqa2}, \ref{Eqa4} se torna:
\begin{align*}
	\ddt{y_3} + 5 \dt{y_3} + 6 y_3 & = a \left( \ddt{y_1} + 5 \dt{y_1} + 6 y_1 \right) + b \left( \ddt{y_2} + 5 \dt{y_2} + 6 y_2 \right) \\
	& = \ddt{}(a y_1 + b y_2) + 5 \dt{}(a y_1 + b y_2) + 6 (a y_1 + b y_2) \\
	& \implies y_3 = a y_1 + b y_2
\end{align*}
Ou seja, a resposta do sistema a uma combina��o linear de diferentes sinais de entrada � a combina��o linear das respostas a cada um desses sinais. \\

\hlin

b) \textbf{Invari�ncia no tempo} \\
A resposta $ y $ a um impulso $ \De{}(t) $ � dada por:
\begin{align}
	\ddt{y} + 5 \dt{y} + 6 y & = 3 \dt{} \De{}(t) + \De{}(t) \notag \\
	& = 3 \dDe{}(t) + \De{}(t) \label{Eqa5}
\end{align}
Seja $ x_1 = \De{}(t - a) $ uma entrada aplicada em $ t = a, \; a \in \mathbb{R} $. A resposta do sistema a essa entrada ser�:
\begin{align*}
	\ddt{y_1} + 5 \dt{y_1} + 6 y_1 & = 3 \dt{x_1} + x_1 \\
	& = 3 \dDe{}(t - a) + \De{}(t - a)
\end{align*}
Fazendo $ \tau = t - a $, ent�o $ d \tau = dt $ e teremos:	
\begin{align}
	\ddtau{y_1} + 5 \dtau{y_1} + 6 y_1 & = 3 \dDe{}(\tau) + \De{}(\tau) \label{Eqa6}
\end{align}
\ref{Eqa6} indica que $ y_1 $ tem a mesma forma de $ y $, estando apenas deslocada no eixo do tempo pela quantidade $ a $. \\

\hlin
\clearpage

c) \textbf{Causalidade} \\
Seja $ x = 0 \; \forall t < a, \; a \in \mathbb{R} $. Nesse caso, considerando-se condi��es iniciais nulas ($ y(0) = \dot{y}(0) = 0 $), a sa�da em $ t < a $ ser�:
\begin{align*}
	\ddt{y} + 5 \dt{y} + 6 y & = 0 \\
	& \implies y = 0 \; \forall t < a
\end{align*}
Assim, o sistema � causal, pois a sa�da � nula antes de o sinal de entrada ser aplicado. \\
\end{resposta}

\clearpage

\begin{pergunta}
\perguntab{}
\end{pergunta}
\begin{resposta}
a) Propriedade \textgoth{P3} (deslocamento na frequ�ncia): \\
Seja $ F(s) = \La{} \{ f(t) \} $ e $ g(t) = e^{\alpha t} f(t), \; \alpha \in \mathbb{Z} $. Ent�o:
\begin{align*}
	\La{} \{ g(t) \} & = \La{} \{ e^{\alpha t} f(t) \} \\ 
	& = \int_0^\infty e^{\alpha t} f(t) e^{-st} \; dt \\
	& = \int_0^\infty f(t) e^{-(s - \alpha)t} \; dt \\
	& = \int_0^\infty f(t) e^{-zt} \; dt \qquad \Aboxed{z = s - a} \\
	& = \left. F(z) \right|_{z = s - a} \\
	& = F(s - a)
\end{align*}

\hlin

b) Propriedade \textgoth{P4} (derivada): \\
Seja $ F(s) = \La{} \{ f(t) \} $ e $ g(t) = \dt{} f(t) $. Ent�o:
\begin{align}
	\La{} \{ g(t) \} & = \La{} \left\{ \dt{} f(t) \right\} \notag \\ 
	& = \int_0^\infty \left[ \dt{} f(t) \right] e^{-st} \; dt \label{Eqb1}
\end{align}
Usando a t�cnica de integra��o por partes:
\begin{align}
	\int u \; dv & = u v - \int v \; du \notag \\
	& \begin{matrix} u & = & e^{-st} & \implies & du & = & -s \; e^{-st} \; dt \\
	                dv & = & \dt{} f(t) \; dt & \implies & v & = & f(t) \end{matrix} \notag \\
	& \wasytherefore \; \begin{matrix} u v & = & e^{-st} f(t) \\
                                       v \; du & = & -s e^{-st} \; f(t) \; dt \end{matrix} \label{Eqb2}
\end{align}
Substituindo \ref{Eqb2} em \ref{Eqb1}, teremos:
\begin{align*}
	\int_0^\infty \left[ \dt{} f(t) \right] e^{-st} \; dt & = \left. \left[ e^{-st} f(t) \right] \right|_0^\infty - \int_0^\infty -s e^{-st} \; f(t) \; dt \\
	& = [0 - f(0)] + s \int_0^\infty e^{-st} \; f(t) \; dt \qquad \Aboxed{\Re \{s\} > 0} \\
	& = s F(s) - f(0)
\end{align*}

\end{resposta}

\end{document}
